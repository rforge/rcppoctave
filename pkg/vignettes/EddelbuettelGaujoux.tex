%\VignetteDepends{RcppOctave,rbenchmark}
%\VignetteCompiler{knitr}

\title{RcppOctave: Octave along with R}  %% just a draft title
\author{by Dirk Eddelbuettel and Renaud Gaujoux}

\maketitle

\abstract{
  The \pkg{RcppOctave} package connects Octave to R. TODO: Expand
}

\section{Introduction}

Octave \citep{Eaton:2008} is a high-level interactive language which is
primarily intended for numerical computations. As it is mostly compatible
with Matlab \citep{MATLAB:2010}, it has found widespread adoption across
different disciplines. While Matlab has historically been pre-eminent in
domains such as electrical engineering and signal processing, it is also
widely used in other fields such as bioinformatics or finance.  Consequently,
a large corpus of application programs are available as \code{.m}-files
(named after the commonly-chose file extension).

R \citep{R:2012}, a language and environment for statistical
computing and graphics, has become the dominant language for statistical
research, and a widely-used environment for empirical work in a variety of
fields.

While both languages share commonalities, their respective focus is different
making a combination of both environments an even more compelling choice.
This short paper illustrates the \pkg{RcppOctave} package by
\cite{CRAN:RcppOctave} which implements an interface between both these
environments.

\section{Octave}

TODO: Two or three short paragraphs about Octave

\section{Example: Matrix Operations}  % or something else

TODO: Show two or three simple call illustrating the variable arguments etc.
\section{Example: Kalman Filter}

\cite{Eddelbuettel+Sanderson:2012} motivate the \pkg{RcppArmadillo} package
by comparing a Kalman Filter implementation in R and C++. As the code underlying
this example was initially published for Matlab\footnote{See
  \url{http://www.mathworks.com/products/matlab-coder/demos.html?file=/products/demos/shipping/coder/coderdemo_kalman_filter.html}.},
it can of course also be used with RcppOctave.

TODO: Few words about example

\begin{smallexample}
function Y = kalmanM(pos)
  dt=1;	
  %% Initialize state transition matrix
  A=[ 1 0 dt 0 0 0;...     % [x  ]
     0 1 0 dt 0 0;...     % [y  ]
     0 0 1 0 dt 0;...     % [Vx]
     0 0 0 1 0 dt;...     % [Vy]
     0 0 0 0 1 0 ;...     % [Ax]
     0 0 0 0 0 1 ];       % [Ay]
  % Initialize measurement matrix
  H = [ 1 0 0 0 0 0; 0 1 0 0 0 0 ];    
  Q = eye(6);
  R = 1000 * eye(2);
  x_est = zeros(6, 1);             
  p_est = zeros(6, 6);

  numPts = size(pos,1);
  Y = zeros(numPts, 2);

  for idx = 1:numPts
    z = pos(idx, :)';
      
    %% Predicted state and covariance
    x_prd = A * x_est;
    p_prd = A * p_est * A' + Q;
    %% Estimation
    S = H * p_prd' * H' + R;
    B = H * p_prd';
    klm_gain = (S \ B)';
    %% Estimated state and covariance
    x_est = x_prd + klm_gain * (z - H * x_prd);
    p_est = p_prd - klm_gain * H * p_prd;
    %% Compute the estimated measurements
    Y(idx, :) = H * x_est;
  end                % of the function
end   % of the function
\end{smallexample}


TODO: More about the example, performance comparison, knocking socks off R


\section{Example: Gibbs Sampler}

Wilkinson (TODO: Reference from his repeated blog posts) used a simple
bivariate Gibbs Sampler as a basis for comparisons between different
programming languages such as C, Java, Python and R. His example has been
re-used in number of other presentations.  

We can adapt this example here as it provides a suitable framework for
showing how \pkg{RcppOctave} can interact with the random number generators
in R. 


\bibliography{../inst/REFERENCES}


\address{Dirk Eddelbuettel \\
  %Affiliation\\
  River Forest, IL\\
  USA}\\
\email{edd@debian.org}

\address{Renaud Gaujoux \\
  Computational Biology \\
  University of Cape Town \\
  Cape Town \\
  South Africa}\\
\email{renaud@cbio.uct.ac.za}

%%% Local Variables: 
%%% mode: latex
%%% TeX-master: "RJwrapper"
%%% End: 
